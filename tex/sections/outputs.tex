\section{Configuring \texorpdfstring{NEMO-$\mathrm{SI}^3$}{NEMO-SI3} Output Files}
\label{sec:outputs}

It is possible to change the content and output frequency of the NEMO-\SIcu{} output files by editing the configuration files \verb|file_def_nemo-oce.xml| and \verb|file_|\linebreak\verb|def_nemo-ice.xml| in the run directory.
The former controls ocean output, the latter sea ice output.
The files are written in XML, which is similar to HTML.
The full story is in Chapter 10 of the NEMO manual, but here is a quick introduction.

I assume that you have got at least as far as running NEMO-\SIcu{} for the first time as in Section~\ref{sec:tutorial:subsec:running-custom}.
Take a look at the file \verb|file_def_nemo-ice.|\linebreak\verb|xml| in the run directory.
The main sea ice output file is the file \verb|ORCA2_5d_|\linebreak\verb|00010101_00011231_icemod.nc| of five-day means, which is defined in \verb|file_|\linebreak\verb|def_nemo-ice.xml| between a pair of lines that look like this:

\begin{verbatim}
  <file id="file21" name_suffix="_icemod" description="ice variables" enabled=".true." >
  ...
  </file>
\end{verbatim}

\noindent{}The variables in the file are defined by intervening lines that look like this:

\begin{verbatim}
  <field field_ref="icethic"          name="sithic" />
\end{verbatim}

\noindent{}You can add further variables by adding more lines of this form.
All the possible output variables are listed in the file \verb|field_def_nemo-ice.xml| (or \verb|field_def_nemo-oce.xml| for the ocean).
The \verb|field_ref| attribute in \verb|file_|\linebreak\verb|def_nemo-ice.xml| should be the same as the \verb|id| attribute in \verb|field_def_nemo_|\linebreak\verb|ice.xml|.
The name attribute in \verb|file_def_nemo-ice.xml| is the name that the variable will have in the netCDF file (NEMO-\SIcu{} is fond of having multiple names for the same thing).

To define a file with a different output frequency, first note that the \verb|<file ...>| \verb|...</file>| lines above are enclosed between a pair of lines that look like this:

\begingroup\small\begin{verbatim}
  <file_group id="5d" output_freq="5d"  output_level="10" enabled=".TRUE.">  <!-- 5d files -->
  ...
  </file_group>
\end{verbatim}\endgroup

\noindent{}(There can be multiple file definitions in a \verb|file_group|.
The group above also contains a file with the suffix \verb|_SBC_scalar|, but this does not get output---I think because the variables in it are only output if certain namelist options are set.)

If you wanted to output daily means instead of five-day means, you could just change all the \verb|5d|'s to \verb|1d|'s in the \verb|file_group| line above.
If you wanted to output daily means as well as five-day means, you could add a new \verb|file_group| by copying the whole \verb|5d| \verb|file_group| (everything between \verb|<file_group ...>| and \verb|</file_group>|) and changing \verb|5d| to \verb|1d|.
In this case you would also need new \verb|id| attributes for the new files.
These attributes should have the form \verb|"fileN"| where \verb|N| is an integer between 1 and 999 and should be different for different files.
For the standard five-day file, the \verb|id| is \verb|"file21"|.
I just increment the number at the end for the files that I add, starting from the highest number currently in \verb|file_def_nemo-ice.xml|.

If you wanted to output monthly means, there is already a suitable \verb|file_group| with a commented-out file definition.
You could add your own \verb|file| definition here.

There are also empty \verb|file_groups| for various periods between one timestep and ten years.
To add \verb|file| definitions to these, you would have to change the \verb|/>| to \verb|>| and end the \verb|file_group| with a line that says \verb|</file_group>| after the file definitions.

If you wish to write output at every timestep, be aware that sea ice processes can have a different timestep to ocean processes, with the ratio between the two controlled by namelist variable \verb|nn_fsbc| in namelist group \verb|namsbc| of \verb|namelist_cfg| (see Section \ref{sec:mods:subsec:running}).
I have found that having two different timesteps can confuse the output code of NEMO-\SIcu{} (or the XIOS library that it uses for output), with the result that the time coordinate in files with an output frequency of one timestep is wrong.
This is not a problem if the two timesteps are the same, which is another reason for running with \verb|nn_fsbc| = 1 as recommended in Section~\ref{sec:mods:subsec:running}.

For experiments running for more than a year of simulation time, I generally split the output files into separate files for each year (that way something useful can be salvaged if the run ends prematurely).
If you want to do this, you need to edit \verb|file_def_nemo-ice.xml| and \verb|file_def_nemo-oce.xml| to add \verb|split_freq="1y"| to the \verb|file_group| definitions.
So, for example, the start of the five-day \verb|file_group| in \verb|file_def_nemo-ice.xml| would become

\begingroup
    \small
    \begin{verbatim}
  <file_group id="5d" output_freq="5d" split_freq="1y" output_level="10" enabled=".TRUE.">
    \end{verbatim}
\endgroup
