\section{Compiling and Running with Local Modifications to the Source}
\label{sec:mods}

This section describes how to to compile and run a version of NEMO-\SIcu{} that incorporates my bug fixes and other modifications to the topographic melt pond scheme.

\subsection{Management of Local Modifications}
\label{sec:mods:subsec:management}

There are two ways to manage local modifications to the NEMO-\SIcu{} source code.
One way is to directly edit the files under the top-level source directory.
This way has the disadvantage that it can create problems when you wish to update the downloaded source from the official repository.
These problems are easier to solve with Git than with many other revision control systems because Git is by design a distributed system and has the ability to create local branches of the source code without requiring write access to the official repository.
However, NEMO has only used Git since Mar 2022, and the local modifications to the topographic melt pond scheme predate that.

The NEMO-\SIcu{} build process provides an alternative way to manage local modifications to the source code.
A configuration directory such as \verb|cfgs/| \verb|UOR_ORCA2_ICE| contains a subdirectory called \verb|MY_SRC|.
Any Fortran source files placed in this directory will be compiled when the configuration is compiled, and will take priority over any similarly named files in the official source.
Local modifications can therefore be maintained outside of the top-level source directory using any revision control system or none.
When the time comes to compile a configuration, copies of these files (or links to them) are placed in the configuration's \verb|MY_SRC| directory.
This is the procedure used for the local modifications to the topographic melt pond scheme.

The local modifications to NEMO-\SIcu{} and XIOS are maintained in the Subversion repository \verb|SI3RACC/Software/SVN|.
Subversion was used because that is what NEMO used until Mar 2022.
A copy of the NEMO-\SIcu{} modifications, checked out and ready for use, can be found under \verb|SI3RACC/Software/| \verb|mods-trunk|.

If you do not wish to access the repository directly, you can skip this paragraph.
If you would like to access the repository directly, you can do so using the URL \verb|file:///storage/silver/cpom/rk901985/SI3RACC/Software/SVN| in Subversion commands.
For example, to check out your own copy of the modifications to NEMO-\SIcu{} (perhaps as a prelude to further changes), use the following command:

\begin{verbatim}
  svn co file:///storage/silver/cpom/rk901985/SI3RACC/Software/SVN/mods/trunk mods-trunk
\end{verbatim}

\noindent{}Warning: The repository is incompatible with the version of Subversion currently on the RACC, which is rather old (1.7.14).
It works fine with the newer version on the NX servers (1.13.0).
Thus, even though the checked-out modifications are compiled on the RACC, operations on the repository have to be carried out on one of the NX servers or another machine with an up-to-date Subversion.
None of this matters if you are using the modifications that have already been checked out under \verb|SI3RACC/Software/mods-trunk|.


\subsection{Compiling}
\label{sec:mods:subsec:compiling}

To recompile the \verb|UOR_ORCA2_ICE| configuration to include the melt pond modifications, copy the four files under \verb|SI3RACC/Software/mods-trunk/src/ICE| to \verb|cfgs/UOR_ORCA2_ICE/MY_SRC| under the top-level source directory.
Next, make sure that the \verb|gcc/6.4.0| and \verb|netcdf| modules are loaded as in Section~\ref{sec:tutorial:subsec:compiling}.
Finally, give the following command in the top-level source directory:

\begin{verbatim}
  ./makenemo -r UOR_ORCA2_ICE
\end{verbatim}

\noindent{}(The more complicated command of Section~\ref{sec:tutorial:subsec:no-pisces} is not needed now that the \verb|UOR_ORCA2_ICE| configuration has been first set up.)
The above command does an intelligent recompilation that only recompiles those parts of NEMO-\SIcu{} that are contained in the new files or depend on the new files (although this happens to include a large part of the code because one of the modified files is \verb|ice.F90|, which defines many of the ice variables).


\subsection{Updating the Template Run Directory}
\label{sec:mods:subsec:run-directory}

Before the new version of NEMO-\SIcu{} can be run, some changes must be made to the template run directory \verb|cfgs/UOR_ORCA2_ICE/EXP00|, so change to this directory.
First, the new code defines some new namelist variables in \verb|namelist_ice_|\linebreak\verb|ref|, so remove the existing file of this name and replace it by a copy of \verb|SI3RACC/| \verb|Software/mods-trunk/cfgs/SHARED/namelist_ice_ref|.
Next, the new code adds a field to the list of all possible ice output fields in \verb|field_def_nemo-ice.|\linebreak\verb|xml|, so remove the existing file and replace it by a copy of \verb|SI3RACC/Software/| \verb|mods-trunk/cfgs/SHARED/field_def_nemo-ice.xml|.
This completes the updating of \verb|EXP00| as a template run directory.

\subsection{Final Configuration and Running}
\label{sec:mods:subsec:running}

The template \verb|EXP00| can now be used to create a run directory for a second experiment:

\begin{verbatim}
  cp -a EXP00 2
\end{verbatim}

\noindent{}As before, you can use a name other that 2 if you wish.
There is a job script suitable for running this experiment named \verb|doit2| in \verb|SI3RACC/Experiments/2| (in fact this script is identical to the previous \verb|doit1|).
Copy this script to the run directory; again, you can change its name if you wish.

It would be possible to run the experiment at this point, but the results would not be very different to the first experiment because the default values in the new \verb|namelist_ice_ref| leave the new melt pond features turned off.
This is part of a general policy of having NEMO-\SIcu{} revert to its standard behaviour by default.
There are further files in \verb|SI3RACC/Experiments/2| that should be copied to the run directory to give more interesting results. These files are as follows:

\begin{description}

    \item[\texttt{namelist\_cfg}]
        This specifies the run-specific namelist variables for the ocean model.
        In namelist group \verb|namsbc|, the namelist variable \verb|nn_fsbc| has been set to 1 instead of 4.
        This variable controls the frequency with which the surface boundary condition module (which includes the sea ice processes) is called.
        The ocean timestep of this experiment is 90\,min (set by namelist variable \verb|rn_Dt| in namelist group \verb|namdom|).
        A value of 4 for \verb|nn_fsbc| means that the surface boundary condition timestep is four times this, or 6\,h, which is rather large.
        Decreasing \verb|nn_fsbc| to 1 decreases the surface boundary condition timestep to 90\,min.
        This value is recommended for all experiments.

    \item[\texttt{namelist\_ice\_cfg}]
        This specifies the run-specific namelist variables for \SIcu{}.
        There are two sets of changes in this file.

        In namelist group \verb|namthd_pnd|, the topographic melt pond scheme has been activated instead of the default level melt pond scheme.
        All the local modifications to the topographic scheme have also been activated.
        See Appendix \ref{sec:topo-nml:subsec:ice-ref} for a list of the namelist variables controlling these modifications.

        In namelist group \verb|namalb|, the namelist variable \verb|rn_alb_hpiv| has been set to 0.06 instead of its default value of 1.
        \SIcu{} reduces the albedo of all ice thinner than this value (in m).
        This makes sense when melt ponds are not being explicitly modelled, but makes less sense when they are.
        It is not possible to explicitly turn this feature off, but reducing the value of \verb|rn_alb_hpiv| reduces the area of ice affected.
        Furthermore, \SIcu{} has a minimum sea ice thickness set by namelist variable \verb|rn_himin| in namelist group \verb|namitd|.
        This has a default value of 0.1.
        Setting \verb|rn_alb_hpiv| to a value below this (as here) has the effect of turning off the reduction of albedo for thin ice completely.
        (The situation is slightly more complicated than this, but the complication only arises if \verb|rn_himin| has a value of 0.05 or less.)

        The above changes to \verb|namelist_ice_cfg| are recommended for all work with the topographic melt pond scheme.

    \item[\texttt{file\_def\_nemo-oce.xml}]
        This specifies what is in the ocean output files (see Appendix \ref{sec:outputs}).
        Some heat flux variables that have proved useful for diagnostic purposes have been added.

    \item[\texttt{file\_def\_nemo-ice.xml}]
        This specifies what is in the ice output files (see Appendix \ref{sec:outputs}).
        A full set of melt pond variables has been added, as have ice albedo and some heat and mass flux variables that have proved useful for diagnostic purposes.

\end{description}

Copy the above files from \verb|SI3RACC/Experiments/2| to the run directory, allowing the files already there to be overwritten.
The experiment can then be submitted to the batch system using the following command in the run directory:

\begin{verbatim}
  sbatch ./doit2
\end{verbatim}
