\section{Using Restart Files}
\label{sec:restarts}

This section describes how to run NEMO-\SIcu{} with initial conditions taken from restart files saved by a previous run.
As an example, we shall extend the three-year run of Section~\ref{sec:forcings} for another three years.

If you look in the run directory of experiment 3 of Section~\ref{sec:forcings}, as well as the output files previously listed, you will see several files with names beginning \verb|ORCA2_00017520_restart|.
These are restart files written at the final timestep, which was timestep 17520.

Start by configuring a new run directory in the same way as for the original experiment in Section~\ref{sec:forcings}, but with links to the CIAF files for 2003--2005 rather than 2000--2002.
You can call this new directory anything you like, but if simply extending an existing run like this, I adopt the following convention: if the original run directory was called \verb|dir|, I call the run directory for the first extension \verb|dir-1| with \verb|dir-2|, \verb|dir-3|, \ldots{} for any extensions after that.

Now make the following edits in the new run directory.
(As noted previously, when adding new variables to \verb|namelist_cfg| or \verb|namelist_ice_cfg| I generally copy and paste the appropriate lines from \verb|namelist_ref| or \verb|namelist_ice_ref| as a starting point.)

In \verb|namelist_cfg|, make the following changes and additions in namelist group \verb|namrun|:

\begin{verbatim}
  nn_it000 = 17521 [the old nn_itend + 1]
  nn_itend = 35040 [the total number of timesteps in six years]
  ln_rstart = .true.
  nn_rstctl = 2
  cn_ocerst_in = "ORCA2_00017520_restart"
  cn_ocerst_indir = "../3" [relative path to the original run directory]
\end{verbatim}

\noindent{}Look at the comments in \verb|namelist_ref| if you want to know what these variables mean.

The following changes also need to be made in namelist groups \verb|namtsd| and \verb|namberg| if they are present in \verb|namelist_cfg|, as is the case for experiment 3.
The changes can be omitted if these groups are absent, as is the case for experiments derived from the reference configuration \verb|ORCA2_SAS_ICE|, such as those of Sections~\ref{sec:sas} and \ref{sec:mlm}.
In namelist group \verb|namtsd| delete the following:

\begin{verbatim}
  ln_tsd_init
\end{verbatim}

\noindent{}(This is not essential, but doing so will avoid a warning.)
In namelist group \verb|namberg| make the following additions:

\begin{verbatim}
  cn_icbrst_in = "ORCA2_00017520_restart_icb"
  cn_icbrst_indir = "../3" [relative path to the original run directory]
  nn_test_icebergs = 0
\end{verbatim}

In \verb|namelist_ice_cfg|, make the following additions in namelist group \verb|nampar|:

\begin{verbatim}
  cn_icerst_in = "ORCA2_00017520_restart_ice"
  cn_icerst_indir = "../3" [relative path to the original run directory]
\end{verbatim}

\noindent{}And in namelist group \verb|namini|:

\begin{verbatim}
  nn_iceini_file = 2
\end{verbatim}

\noindent{}Look at the comments in \verb|namelist_ice_ref| if you want to know what these variables mean.

For this extension of experiment 3, the job script (which I would call \verb|doit3-1|) can be a copy of \verb|doit3|.
If the extension covered a different number of years, the time limit in the job script might need adjustment, but that is not the case here.
