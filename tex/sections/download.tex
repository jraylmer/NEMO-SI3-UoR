\section{Downloading the Source Code}
\label{sec:download}

The official NEMO-\SIcu{} source code is managed using the Git revision control system.
There is public read access to the source code repository.
To create a local copy of the repository, use the following command:

\begin{verbatim}
  git clone https://forge.nemo-ocean.eu/nemo/nemo.git
\end{verbatim}

\noindent{}This will create a directory called \verb|nemo| under the current directory and download the NEMO-\SIcu{} source to it.
This \verb|nemo| directory will be referred to as the \textit{top-level source directory} in what follows.

The leading edge of NEMO-\SIcu{} development takes place on the so-called \textit{main} branch of its source code repository.
There are also official release versions from time to time.
The source code under the top-level source directory initially corresponds to the latest revision of the main branch.
The instructions in this document have been tested with the main branch as it was on 27/4/2022.
It is recommended that you switch to this revision before trying out the instructions.
This can be done by giving the following command in the top-level source directory:

\begin{verbatim}
  git checkout b9ec6ca14aea39a6101e3fc0cee2e26f69fe88d1
\end{verbatim}

\noindent{}As can be seen, Git labels revisions (or \textit{commits} as it calls them) with 40-digit hexadecimal strings.
After giving this command, the source code under the top-level source directory will correspond to the older revision with which the instructions have been tested.

Later, if you wish to return to the head of the main branch, you can do so with the following command:

\begin{verbatim}
  git checkout main
\end{verbatim}
Alternatively, if you wish to switch to the 4.2.0 release version, you can give the command

\begin{verbatim}
  git checkout 4.2.0
\end{verbatim}
