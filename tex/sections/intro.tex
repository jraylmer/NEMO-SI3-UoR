\section{Introduction}
\label{sec:intro}

This document gives instructions for compiling and running the \SIcu{} sea ice model, which is part of the NEMO ocean modelling framework, on the University of Reading Academic Computing Cluster 2 (RACC2).
It is based on a version originally written by David M. Livings in 2022 for the old RACC, which is now out of service and superseded by RACC2.
The new cluster has different versions of the required compilers and libraries, and there are now also new major releases of \NEMOSIcu{}.
This necessitates the present documentation for running the latest versions of \NEMOSIcu{} on RACC2.

The source \LaTeX{} files are now maintained in a GitHub repository with the initial commit corresponding to David's original document.
Other resources such as configuration files for compilations are also provided in this repository (rather that relying on paths to storage volumes on the cluster itself).
They are referred to throughout as the \textit{auxiliary resources}.

By following the instructions in sections~\ref{sec:xios}--\ref{sec:tutorial}, the reader will end up with a configuration of \SIcu{} coupled to the NEMO ocean model and forced by standard climatological atmospheric forcing.
Updated explanations of how to use multi-year atmospheric forcing, replace the full ocean model with the standalone surface scheme (SAS) or mixed-layer model, and how to use restart files, will be added back to this document later.

The main technical website for NEMO is its GitLab site at:

\begin{center}
    \url{https://forge.nemo-ocean.eu}
\end{center}

Here, the source code of \NEMOSIcu{} can be browsed and a link can be found to a manual for the ocean part of NEMO \citep{nemo-5.0.0}, which corresponds to version 5.0.
Currently, the latest available \SIcu{} manual is that for version 4.2 \citep{si3-4.2.0}, but the release notes for version 5.0 can be found at:

\begin{center}
    \url{https://forge.nemo-ocean.eu/nemo/nemo/-/releases/5.0}
\end{center}
