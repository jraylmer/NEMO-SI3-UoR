\section{Compiling and running a first experiment}
\label{sec:tutorial}

\subsection{Compiling reference configuration \texorpdfstring{\texttt{ORCA2\_ICE\_PISCES}}{ORCA2 ICE PISCES}}
\label{sec:tutorial:subsec:compiling}

Compilation of \NEMOSIcu{} is done with the script \verb|makenemo| (no extension) in the top-level source directory.
Platform-specific compiler options are specified in a configuration file in the \verb|arch| subdirectory.
This is a similar workflow to XIOS (section~\ref{sec:xios}), but for NEMO there is only one configuration file required; that for compiling on RACC2 can be found in the auxiliary resources under \verb|nemo/arch/arch-racc2.fcm|.

\begin{enumerate}
    \item Make a copy of this file under the \verb|arch| subdirectory of your top-level NEMO source directory.
\end{enumerate}

For best results, \NEMOSIcu{} should be compiled to use the XIOS library for input and output as described in section~\ref{sec:xios}.
Assuming that has been done, your copy of the configuration file for NEMO must be edited so that NEMO knows where to find your compiled XIOS library:

\begin{enumerate}\setcounter{enumi}{1}
    \item Open your \verb|arch-racc2.fcm| file and, if required, change the value of the parameter \verb|%XIOS_HOME| to the path to your XIOS installation.
    
    The placeholder for this, \verb|${HOME}/software/XIOS2|, corresponds to the location of XIOS2 used in section~\ref{sec:xios}, so if you used the same path then this step can be skipped.
\end{enumerate}

To compile \NEMOSIcu{} requires selecting a \textit{configuration} of model components (and other pre-compilation options).
A set of reference configurations is provided under the \verb|cfgs| subdirectory of the top-level source directory.
To begin, we shall compile the \verb|ORCA2_ICE_PISCES| reference configuration.
This includes the full NEMO ocean model on the ORCA2 global grid, the \SIcu{} sea ice model, and the PISCES biogeochemical model.
We shall not run this configuration because PISCES turns out to be a resource hog and is not needed for most sea ice work.
However, compiling this standard configuration will act as a check that the build machinery is working properly, as well as introduce the basic compilation command, before we get on to custom configurations in section~\ref{sec:tutorial:subsec:no-pisces}.

Before running the compilation script \verb|makenemo|, it is necessary to prepare the environment by loading software modules.
For XIOS this was done automatically via the extra configuration file \verb|xios2/arch-racc2.env|; NEMO does not have such an interface with its compilation script so we must do it manually:

\begin{enumerate}\setcounter{enumi}{2}
    \item Load the required environment modules by giving the command:

\begin{verbatim}
  $ module purge
  $ module load mpi/openmpi-x86_64
\end{verbatim}
\end{enumerate}

It is recommended to first use \verb|module purge| to clear the environment of any other loaded modules because other modules have been known to somehow interfere with the compilation.

\begin{enumerate}\setcounter{enumi}{3}
    \item Compile the \verb|ORCA2_ICE_PISCES| configuration with the following command:
\begin{verbatim}
  $ ./makenemo -m racc2 -r ORCA2_ICE_PISCES
\end{verbatim}
\end{enumerate}

Compilation will take a few minutes, during which time \verb|makenemo| gives a commentary on what it is doing.
If all goes well, the final lines should be something like:

\begingroup\small
\begin{verbatim}
------------------------------------------------------------------------
Compilation successful
------------------------------------------------------------------------

"./nemo" -> "../BLD/bin/nemo.exe"

------------------------------------------------------------------------
Created symbolic link ./nemo to NEMO executable in experiment directory
${HOME}/nemo/cfgs/ORCA2_ICE_PISCES/EXP00/nemo
------------------------------------------------------------------------
\end{verbatim}
\endgroup


\subsection{Compiling a custom configuration without PISCES}
\label{sec:tutorial:subsec:no-pisces}

As noted in Section \ref{sec:tutorial:subsec:compiling}, it is not recommended to run \SIcu{} with the PISCES biogeochemical model.
It is possible to create custom configurations of \NEMOSIcu{} based on the reference configurations, and we shall create one by omitting PISCES from \verb|ORCA2_ICE_PISCES|.
We need a name for our configuration.
An obvious choice is \verb|ORCA2_ICE|, but it is helpful to distinguish between custom and the reference configurations (e.g., to avoid a possible future name clash with a new reference configuration).
The following instructions will call the new configuration \verb|UOR_ORCA2_ICE|.

To compile the custom configuration, first make sure that the necessary modules are loaded by repeating step~3 in section \ref{sec:tutorial:subsec:compiling} if necessary (which it will not be if this is part of the same login session).
The compilation command is (note the following is one single command---do not include the \verb|...|):

\begingroup
\begin{verbatim}
  $ ./makenemo -m racc2 -r ORCA2_ICE_PISCES -n UOR_ORCA2_ICE ...
  ...          -d "OCE ICE NST ABL" del_key key_top
\end{verbatim}
\endgroup

Give this command, and whilst it is running you can read the following explanation of what it means.

The \verb|-m| and \verb|-r| flags are the same as before, specifying the \textbf{m}achine architecture file and \textbf{r}eference configuration respectively.
Now we have added the \verb|-n| flag, which specifies the name of the \textbf{n}ew configuration.

Configurations are made up of sub-components that correspond to subdirectories of the \verb|src| subdirectory of the top-level source directory.
The sub-components of the reference configurations are listed in the file \verb|cfgs/ref_cfgs.txt|.
For \verb|ORCA2_ICE_PISCES| they are \verb|OCE|, \verb|TOP|, \verb|ICE|, \verb|NST|, and \verb|ABL|.
TOP stands for `Tracers in Ocean Paradigm' and is the part of NEMO containing PISCES.
This is the sub-component that we wish to omit.
The \verb|-d| option is used to specify the sub-components to be used in the new configuration.
The command above specifies all the sub-components in \verb|ORCA2_ICE_PISCES| except for \verb|TOP|.

Finer-grained control over which parts of \NEMOSIcu{} are compiled is provided by a set of preprocessor keys, which control whether certain sections of individual source files are included or omitted.
The keys for \verb|ORCA2_ICE_PISCES| are listed in the file \verb|cfgs/ORCA2_ICE_PISCES/cpp_ORCA2_ICE_PISCES.fcm|.
They include \verb|key_si3|, which includes the \SIcu{} code (we want to keep this), and \verb|key_top|, which includes the TOP code (we want to get rid of this).
The arguments \verb|del_key key_top| to the above command delete \verb|key_top| when creating the keys for \verb|UOR_ORCA2_ICE| from those for \verb|ORCA2_ICE_PISCES|.


\subsection{Preparing a template run directory}
\label{sec:tutorial:subsec:run-directory}

Compiling the custom configuration \verb|UOR_ORCA2_ICE| will result in a new directory \verb|cfgs/UOR_ORCA2_ICE| under the top-level source directory.
This directory has various subdirectories, of which the most important is \verb|EXP00|.
This contains a symbolic link called \verb|nemo| to the \NEMOSIcu{} executable and various configuration files.
The files with names beginning \verb|namelist_| are Fortran namelist files that specify run-time options.
The namelist files with names ending \verb|_ref| contain the defaults and should not be edited.
Instead, changes to the defaults should be made in the namelist files with names ending \verb|_cfg|.
Settings in these files will override the defaults.
The other configuration files all have names ending \verb|.xml|.
These are XML files used to configure XIOS.
In particular, the XML files with names beginning \verb|file_| can be edited to change what is included in the output files from \NEMOSIcu{}.

It would be possible to run \NEMOSIcu{} in \verb|EXP00|, but since it is common to run the same executable multiple times with different configuration settings, it is better to keep \verb|EXP00| as a template from which additional run directories can be created as needed.
Before our new \verb|EXP00| can be used in this manner, some further work is needed.

\begin{enumerate}
    \item The rest of this section assumes that the reader is working in\newline\verb|cfgs/UOR_ORCA2_ICE/EXP00|, so change to this directory.
\end{enumerate}

Because \verb|UOR_ORCA2_ICE| was created from \verb|ORCA2_ICE_PISCES|, its \verb|EXP00| directory contains some redundant configuration files that are only relevant to PISCES.
Deleting the following six files will reduce clutter:

\begin{verbatim}
namelist_pisces_cfg        namelist_top_ref
namelist_pisces_ref        field_def_nemo-pisces.xml
namelist_top_cfg           file_def_nemo-pisces.xml
\end{verbatim}

\begin{enumerate}\setcounter{enumi}{1}
    \item Delete the above files (e.g., using \verb|rm ./*pisces* ./*_top_*|).
    \item In \verb|context_nemo.xml|, remove the lines referring to the two \verb|.xml| files above.
\end{enumerate}

Some \NEMOSIcu{} configurations require extra input files to be downloaded from:

\begin{center}
    \url{https://gws-access.jasmin.ac.uk/public/nemo/sette\_inputs/}
\end{center}

These files contain things such as ocean climatological fields and atmospheric forcing data.
To run \verb|UOR_ORCA2_ICE|, the required files are those in the package \verb|ORCA2_ICE_v5.0.0.tar.gz|.
You can download and unpack this yourself, but the data is already available on the shared storage (the location is given in step 4 below).
You could copy the files to your \verb|EXP00| directory, but that would waste disk space, particularly if the template is to be duplicated many times.
It is better to just make symbolic links.

\begin{enumerate}\setcounter{enumi}{3}
    \item Link the input files in your \verb|EXP00| directory:

\begingroup\small
\begin{verbatim}
$ ln -s /storage/research/cpom1/NEMO/inputs/sette/ORCA2_ICE_v5.0.0/* .
\end{verbatim}
\endgroup

    Do not miss off the dot at the end.

\end{enumerate}

It is also worth making some namelist (run-time parameters) changes.
There is a file called \verb|namelist_ref| (actually a symbolic link) and one called \verb|namelist_cfg| (not a link).
The former contains all the default namelist parameters, while any written in the latter take precedence.
There is a similar pair of namelist files for \SIcu{}: \verb|namelist_ice_ref| and \verb|namelist_ice_cfg|.
We always edit the \verb|_cfg| namelist files and leave the \verb|_ref| ones alone.

Edit \verb|namelist_cfg| as follows:

\begin{enumerate}\setcounter{enumi}{4}

    \item\textbf{Set the simulation to run for one year.}
        Run lengths are determined by the model timestep, parameter \verb|rn_Dt| in namelist group \verb|&namdom|, and the start and end iteration numbers, parameters \verb|nn_it000| and \verb|nn_itend| respectively in namelist group \verb|&namrun|.
        The default timestep is 10800~s (3~hours), default \verb|nn_it000|~=~1, and default \verb|nn_itend|~=~5840.
        This corresponds to a total run length of 2~years.
        
        For our test runs 1~year is sufficient, so change \verb|nn_itend| to 2920.

    \item\textbf{Deactivate icebergs.}
        Although icebergs are deactivated by default in \verb|namelist_ref|, strangely they are activated by the \verb|namelist_cfg| that is packaged with the \verb|ORCA2_ICE_PISCES| configuration and hence inherited by our custom configuration.
        We generally do not need to activate icebergs so to avoid issues it is best to deactivate them.
        
        Find the parameter \verb|ln_icebergs| in namelist group \verb|&namberg| and change it from \verb|.true.| to \verb|.false.|.
        Note the surrounding dots: this is how booleans are written in Fortran.
        
        Alternatively, delete the line containing \verb|ln_icebergs| or the entire namelist group \verb|&namberg|.

\end{enumerate}

Finally, a job submission script is needed and one is provided in the auxiliary resources:  \verb|nemo/exps/runjob.sh|.
This is explained in section~\ref{sec:tutorial:subsec:running-custom}; for now:

\begin{enumerate}\setcounter{enumi}{6}
    \item Make a copy of the \verb|runjob.sh| script into your template run directory.
\end{enumerate}


\subsection{Running the custom configuration}
\label{sec:tutorial:subsec:running-custom}

The directory \verb|EXP00| can now be used as a template from which to create a directory in which to run \NEMOSIcu{}.
This run directory can be called whatever you like, but it must be located under \verb|cfgs/UOR_ORCA2_ICE| under the top-level source directory.
Here, we will simply call it \verb|EXP01|. 

\begin{enumerate}
    \item Create a new run directory by copying the template directory:

\begin{verbatim}
  $ cp -a ./EXP00 ./EXP01
\end{verbatim}

    Substitute \verb|EXP01| for something else if you prefer.

    \item Change into the new run directory:

\begin{verbatim}
  $ cd ./EXP01
\end{verbatim}

\end{enumerate}

The script \verb|runjob.sh| added in step~7 of section~\ref{sec:tutorial:subsec:run-directory} is used to submit a job to run \NEMOSIcu{} on the compute nodes on RACC2.
The starting point for this script was the example of a distributed memory batch job at:

\begin{center}
    \small
    \url{https://research.reading.ac.uk/act/knowledgebase/academic-cluster-usage/}
\end{center}

The job is configured to run on eight processors with 16~GB of memory (in total) and a time limit of 30~minutes.

\begin{enumerate}\setcounter{enumi}{2}
    \item Submit the job to the batch system using the command:

\begin{verbatim}
  $ sbatch ./runjob.sh
\end{verbatim}

\end{enumerate}

The job status can be checked using Slurm commands (see above link) such as:

\begin{verbatim}
  $ squeue -u $USER
\end{verbatim}

This simulation should take about \mbox{10--15}~minutes to complete.
When it has completed, there will be a number of output files.
The Slurm log file will be called \verb|<x>-<j>.out| where \verb|<x>| is the job name specified in \verb|runjob.sh| parameter \verb|--job-name|, and \verb|<j>| is the Slurm job ID.
More informative is the log file from \NEMOSIcu{}, which is always called \verb|ocean.output|.
Look through this for any errors (try searching for the string `E R R' with spaces between the letters).
The output fields are the following netCDF files (use \verb|$ ls ORCA2*.nc| to check):

\begingroup\small
\begin{verbatim}
  ORCA2_1m_00010101_00011231_diaptr2D.nc
  ORCA2_1m_00010101_00011231_diaptr3D.nc
  ORCA2_5d_00010101_00011231_grid_T.nc
  ORCA2_5d_00010101_00011231_grid_U.nc
  ORCA2_5d_00010101_00011231_grid_V.nc
  ORCA2_5d_00010101_00011231_grid_W.nc
  ORCA2_5d_00010101_00011231_icemod.nc
  ORCA2_5d_00010101_00011231_SBC_scalar.nc
  ORCA2_5d_00010101_00011231_scalar.nc
\end{verbatim}
\endgroup

The file ending \verb|_icemod.nc| contains five-day means of sea ice variables; the others contain ocean variables.
