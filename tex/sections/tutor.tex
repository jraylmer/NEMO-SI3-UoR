\section{Compiling and Running a First Experiment}
\label{sec:tutorial}

\subsection{Compiling the Reference Configuration \texorpdfstring{\texttt{ORCA2\_ICE\_PISCES}}{ORCA2 ICE PISCES}}
\label{sec:tutorial:subsec:compiling}

NEMO-\SIcu{} is compiled using the script \verb|makenemo| in the top-level source directory.
Platform-specific compiler options are specified in a configuration file in the \verb|arch| subdirectory.
A configuration file for compiling on the RACC has been created as \verb|arch-racc_gfortran.fcm| and can be found under \verb|SI3RACC/| \verb|Software/mods-trunk/arch|.
\textit{Make a copy of this file under the arch subdirectory of the top-level source directory.}

For best results, NEMO-\SIcu{} should be compiled to use the XIOS library for input and output.
Whilst it is possible to compile and run without XIOS, this limits the available output options.
The configuration file \verb|arch-racc_| \verb|gfortran.fcm| makes use of the pre-compiled XIOS library under \verb|SI3RACC/| \verb|Software/xios-trunk|.
(Should it ever be necessary to recompile this library, refer to the instructions in Appendix \ref{sec:xios}.)
If you really do want to compile without XIOS, you can use the alternative configuration file \verb|arch-racc_gfortran_| \verb|noxios.fcm|, but this has not been tested recently.

NEMO-\SIcu{} can be compiled in several different configurations.
A set of reference configurations is provided under the \verb|cfgs| subdirectory of the top-level source directory.
We shall compile the \verb|ORCA2_ICE_PISCES| configuration.
This includes the full NEMO ocean model on the ORCA2 global grid, the \SIcu{} sea ice model, and the PISCES biogeochemical model.
We shall not run this configuration because PISCES turns out to be a resource hog and is not needed for most sea ice work.
However, compiling this standard configuration will act as a check that the build machinery is working properly before we get on to custom configurations.

Before running \verb|makenemo|, it is necessary to load the modules \verb|gcc/6.4.0|\footnote{
    If using \texttt{arch-racc\_gfortran\_noxios.fcm} to compile without XIOS, the module \texttt{gcc/6.} \texttt{4.0} should not be loaded.
    Instead, the default GCC on the RACC (4.8.5) should be used.
    Furthermore, when compiling with XIOS, there is actually some juggling of compiler versions that goes on behind the scenes in \texttt{arch-racc\_gfortran.fcm}.
    The interested reader is referred to the comments at the start of that file for details.
}, \verb|netcdf/4.6.1|, \verb|hdf5/1.8.20|, and \verb|MPI/mpich/gcc/3.2.1| (loading \verb|netcdf/4.| \verb|6.1| will automatically load the last two).
The necessary modules can be loaded with the command

\begin{verbatim}
  module load gcc/6.4.0 netcdf
\end{verbatim}

\noindent{}Give this command if the modules are not already loaded.

NEMO-\SIcu{} can now be compiled by giving the following command in the top-level source directory:

\begin{verbatim}
  ./makenemo -m racc_gfortran -r ORCA2_ICE_PISCES
\end{verbatim}

\noindent{}Compilation will take a few minutes, during which time \verb|makenemo| gives a commentary on what it is doing.
If all goes well, the final lines should be something like:

\begin{verbatim}
  fcm_internal load:F nemo nemo.o nemo.exe
  ar: creating ...
  mpif90 -o nemo.exe ...
  ->Make: 446 seconds
  ->TOTAL: 476 seconds
  Build command finished on Wed May 25 19:53:38 2022.
  /storage/silver/cpom/rk901985/tmp/nemo/cfgs
\end{verbatim}

\noindent{}where the \verb|...|'s will be long lists of files and arguments.


\subsection{Compiling a Custom Configuration without PISCES}
\label{sec:tutorial:subsec:no-pisces}

As noted in Section \ref{sec:tutorial:subsec:compiling}, it is not recommended to run \SIcu{} with the PISCES biogeochemical model.
It is possible to create custom configurations of NEMO-\SIcu{} based on the reference configurations, and we shall create one by omitting PISCES from \verb|ORCA2_ICE_PISCES|.
We need a name for our configuration.
An obvious choice is \verb|ORCA2_ICE|, but I like to prefix my configurations with my initials (\verb|DML_|) so that it is clear which configurations I have created myself and also to avoid a possible future name clash with a new reference configuration.
You may like to adopt a similar policy.
The following instructions will call the new configuration \verb|UOR_ORCA2_ICE|.

To compile the custom configuration, first make sure that the necessary modules are loaded by repeating the \verb|module load| command from Section \ref{sec:tutorial:subsec:compiling} if necessary (which it will not be if this is part of the same login session).
The compilation command is

\begin{verbatim}
  ./makenemo -r ORCA2_ICE_PISCES -n UOR_ORCA2_ICE -d OCE ICE NST ABL del_key key_top
\end{verbatim}

\noindent{}Give this command, and whilst it is running you can read the following explanation of what it means.

The \verb|-r| option specifies the name of the reference configuration on which the new configuration is based.
The \verb|-n| option specifies the name of the new configuration.

Configurations are made up of sub-components that correspond to subdirectories of the \verb|src| subdirectory of the top-level source directory.
The sub-components of the reference configurations are listed in the file \verb|cfgs/ref_cfgs.| \verb|txt|.
For \verb|ORCA2_ICE_PISCES| they are \verb|OCE|, \verb|TOP|, \verb|ICE|, \verb|NST|, and \verb|ABL|.
TOP stands for `Tracers in Ocean Paradigm' and is the part of NEMO containing PISCES.
This is the sub-component that we wish to omit.
The \verb|-d| option is used to specify the sub-components to be used in the new configuration.
The command above specifies all the sub-components in \verb|ORCA2_ICE_PISCES| except for \verb|TOP|.

Finer-grained control over which parts of NEMO-\SIcu{} are compiled is provided by a set of preprocessor keys, which control whether certain sections of individual source files are included or omitted.
The keys for \verb|ORCA2_ICE_|\linebreak\verb|PISCES| are listed in the file \verb|cfgs/ORCA2_ICE_PISCES/cpp_ORCA2_ICE_PISCES.| \verb|fcm|.
They include \verb|key_si3|, which includes the \SIcu{} code (we want to keep this), and \verb|key_top|, which includes the TOP code (we want to get rid of this).
The arguments \verb|del_key key_top| to the above command delete \verb|key_top| when creating the keys for \verb|UOR_ORCA2_ICE| from those for \verb|ORCA2_ICE_PISCES|.

The reader may have noticed that the option \verb|-m racc_gfortran|, used to specify the computing architecture in Section \ref{sec:tutorial:subsec:compiling}, is not included in the command above.
This is because \verb|makenemo| remembers the last architecture configuration file used (in the file \verb|mk/arch.history|) and specifying the \verb|-m| option is unnecessary if the architecture is unchanged from last time.


\subsection{Preparing a Template Run Directory}
\label{sec:tutorial:subsec:run-directory}

Compiling the custom configuration \verb|UOR_ORCA2_ICE| will result in a new directory \verb|cfgs/UOR_ORCA2_ICE| under the top-level source directory.
This directory has various subdirectories, of which the most important is \verb|EXP00|.
This contains a symbolic link called \verb|nemo| to the NEMO-\SIcu{} executable and various configuration files.
The files with names beginning \verb|namelist_| are Fortran namelist files that specify run-time options.
The namelist files with names ending \verb|_ref| contain the defaults and should not be edited.
Instead, changes to the defaults should be made in the namelist files with names ending \verb|_cfg|.
Settings in these files will override the defaults.
The other configuration files all have names ending \verb|.xml|.
These are XML files used to configure XIOS. In particular, the XML files with names beginning \verb|file_| can be edited to change what is included in the output files from NEMO-\SIcu{}.

It would be possible to run NEMO-\SIcu{} in \verb|EXP00|, but since it is common to run the same executable multiple times with different configuration settings, it is better to keep \verb|EXP00| as a template from which run directories can be created as needed.
Before our new \verb|EXP00| can be used in this manner, some further work is needed.
The rest of this section assumes that the reader is working in \verb|cfgs/UOR_ORCA2_ICE/EXP00|, so change to this directory.

Because \verb|UOR_ORCA2_ICE| was created from \verb|ORCA2_ICE_PISCES|, its \verb|EXP00| directory contains some redundant configuration files that are only relevant to PISCES.
Deleting these will reduce clutter.
The files to be deleted are the following:

\begin{verbatim}
  namelist_pisces_cfg
  namelist_pisces_ref
  namelist_top_cfg
  namelist_top_ref
  field_def_nemo-pisces.xml
  file_def_nemo-pisces.xml
\end{verbatim}

\noindent{}The two lines referring to the last two files in \verb|context_nemo.xml| should also be deleted.
You can either make the edit yourself or copy the pre-edited \verb|context_| \verb|nemo.xml| from under \verb|SI3RACC/Experiments/0|.

Some NEMO-\SIcu{} configurations require extra input files to be downloaded from \verb|https://gws-access.jasmin.ac.uk/public/nemo/sette_inputs/|.
These files contain things such as ocean climatological fields and atmospheric forcing data.
To run \verb|UOR_ORCA2_ICE| for a full year, the required files are those in the package \verb|ORCA2_ICE_v4.2_RC_FULL.tar.gz| and one file from \verb|ORCA2_ICE_v4.2.| \verb|0.tar.gz|.
You can either download these packages yourself (for the first package you will need to look under \verb|https://gws-access.jasmin.ac.uk/public/| \verb|nemo/sette_inputs/index_RC.html|) or make use of the pre-downloaded versions under \verb|SI3RACC/Downloads/NEMO|.
More conveniently still, you can use the unpacked versions of the archives under the directories \verb|SI3RACC/Downloads/| \verb|NEMO/ORCA2_ICE_v4.2_RC_FULL| and \verb|SI3RACC/Downloads/NEMO/ORCA2_ICE_v4.| \verb|2.0|.
You could copy the files under these directories to \verb|EXP00|, but that would waste disk space, particularly if the template is to be duplicated many times.
It is better to just make symbolic links with the following commands:

\begin{verbatim}
ln -s /storage/silver/cpom/rk901985/SI3RACC/Downloads/NEMO/ORCA2_ICE_v4.2_RC_FULL/* .
ln -s /storage/silver/cpom/rk901985/SI3RACC/Downloads/NEMO/ORCA2_ICE_v4.2.0/zdfiwm_*.nc .
\end{verbatim}

\noindent{}Do not miss off the dots at the end.


\subsection{Running the Custom Configuration}
\label{sec:tutorial:subsec:running-custom}

The directory \verb|EXP00| can now be used as a template from which to create a directory in which to run NEMO-\SIcu{}.
This run directory can be called whatever you like, but it must be located under \verb|cfgs/UOR_ORCA2_ICE| under the top-level source directory.

I have a policy of numbering my experiments on a project sequentially starting from 1.
If an experiment requires a directory to itself, I just use the experiment number as the directory name.
Thus I would create my first run directory with the command:

\begin{verbatim}
  cp -a EXP00 1
\end{verbatim}

\noindent{}Give this command yourself, substituting something else for \verb|1| if you wish.
Change into the new run directory.

A job script suitable for running our custom configuration of NEMO-\SIcu{} under the batch system on the RACC can be found with the name \verb|doit1| in \verb|SI3RACC/Experiments/1|.
The starting point for this script was the example of a distributed memory batch job at \verb|https://research.reading.ac.| \verb|uk/act/knowledgebase/academic-cluster-usage/|.
The main additions are some lines to load the \verb|gcc| and \verb|netcdf| modules.
The program is run on eight processors with the default memory allocation and a time limit of 16 hours.
Copy \verb|doit1| to the run directory under whatever name you please.
My convention would be to keep the name \verb|doit1|.
The job can then be submitted to the batch system using the command

\begin{verbatim}
  sbatch ./doit1
\end{verbatim}

\noindent{}On a good day, a one-year simulation such as this runs in 30--60 minutes, but I have known jobs to take over 13 hours per simulation year.

When the run has completed, there will be a number of output files.
The log file from the batch system will be called \verb|doit-<N>.out| where \verb|<N>| is the job id.
More informative is the log file from NEMO-\SIcu{}, which is called \verb|ocean.output|.
Look through this for any errors (try searching for the string `E R R' with spaces between the letters).
The principal output fields are five-day means in the following netCDF files:

\begin{verbatim}
  ORCA2_5d_00010101_00011231_icemod.nc
  ORCA2_5d_00010101_00011231_grid_T.nc
  ORCA2_5d_00010101_00011231_grid_U.nc
  ORCA2_5d_00010101_00011231_grid_V.nc
  ORCA2_5d_00010101_00011231_grid_W.nc
\end{verbatim}

\noindent{}The first of these contains sea ice variables, the others contain ocean variables.
