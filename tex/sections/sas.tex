\section{Compiling and Running the Standalone Surface Scheme}
\label{sec:sas}

As an alternative to running the full ocean model, NEMO-\SIcu{} provides a \textit{standalone surface scheme} (SAS) that enables surface processes (including sea ice) to be tested using ocean variables read from files.
This scheme forms the basis for the mixed-layer ocean model that has been implemented locally and is the subject of Section~\ref{sec:mlm}.
Experiments with the standard SAS have shown that it produces sea ice that is much too thick, a problem that is largely fixed by the mixed-layer ocean model.
Nevertheless, this section describes how to compile and run the standard scheme as a prelude to running the mixed-layer model in Section~\ref{sec:mlm}.


\subsection{Compiling}
\label{sec:sas:subsec:compiling}

The reference configuration for the SAS is \verb|ORCA2_SAS_ICE|.
We shall compile a custom version of this called \verb|UOR_ORCA2_SAS_ICE|.
Our custom configuration will incorporate the local modifications to the topographic melt pond scheme and the code for the mixed-layer model.
By default, the mixed-layer model will be turned off, and thus the configuration will run as the standard SAS.

Instead of proceeding as we did with \verb|UOR_ORCA2_ICE|, where we compiled the standard code and then recompiled with the local modifications, we shall proceed directly to compiling with the local modifications.
This means that we have to start by creating the configuration directory and its \verb|MY_SRC| subdirectory.
In the \verb|cfgs| subdirectory of the top-level source directory, give the following command:

\begin{verbatim}
  mkdir -p UOR_ORCA2_SAS_ICE/MY_SRC
\end{verbatim}

\noindent{}Now copy to the new \verb|MY_SRC| directory the four files under \verb|SI3RACC/Software/|\newline\verb|mods-trunk/src/ICE| and the single file under \verb|SI3RACC/Software/mods-trunk/|\newline\verb|src/SAS|.
It is the latter file (\verb|sbcssm.F90|) that includes the mixed-layer model.

Next, make sure that the \verb|gcc/6.4.0| and \verb|netcdf| modules are loaded as in Section~\ref{sec:tutorial:subsec:compiling}.
Finally, compile the configuration by giving the following command in the top-level source directory:

\begin{verbatim}
  ./makenemo -r ORCA2_SAS_ICE -n UOR_ORCA2_SAS_ICE
\end{verbatim}

\noindent{}Since the \verb|ORCA2_SAS_ICE| configuration does not include the PISCES biogeochemical model, we do not need the extra arguments used to remove it in Section~\ref{sec:tutorial:subsec:no-pisces}.


\subsection{Preparing a Template Run Directory}
\label{sec:sas:subsec:run-directory}

As with previous compilations, compilation of \verb|UOR_ORCA2_SAS_ICE| will result in several new subdirectories under \verb|cfgs/UOR_ORCA2_SAS_ICE| under the top-level source directory.
One of these is again \verb|EXP00|, which we shall prepare as a template run directory.
Change to this directory before proceeding.

The first thing to do is to delete \verb|namelist_ref|, \verb|namelist_ice_ref|, and \verb|field_def_nemo-ice.xml| and replace them with copies of the versions under \verb|SI3RACC/Software/mods-trunk/cfgs/SHARED|.
We made a similar replacement of the last two files for \verb|UOR_ORCA2_ICE| in Section \ref{sec:mods:subsec:run-directory}.
The first file (\verb|namelist_ref|) now needs replacing too because it contains some new namelist variables that control the mixed-layer model.

Now we need to add the forcing and other auxiliary files.
As in Section~\ref{sec:tutorial:subsec:run-directory}, we require all the files from the package \verb|ORCA2_ICE_v4.2_RC_FULL.tar.gz| and one file from \verb|ORCA2_ICE_v4.2.0.tar.gz|.
We also require all the files from the package \verb|SAS_v4.2_RC_FULL.tar.gz| (available from the same source as the other packages).
This package contains sample sea surface forcing files for driving the SAS.
As with the other packages, there is a pre-downloaded version in \verb|SI3RACC/Downloads/NEMO| and an unpacked version under \verb|SI3RACC/Downloads|\newline\verb|/NEMO/SAS_v4.2_RC_FULL|.
Create symbolic links to all the required files by giving the following commands in the run directory:

\begin{verbatim}
ln -s /storage/silver/cpom/rk901985/SI3RACC/Downloads/NEMO/ORCA2_ICE_v4.2_RC_FULL/* .
ln -s /storage/silver/cpom/rk901985/SI3RACC/Downloads/NEMO/ORCA2_ICE_v4.2.0/zdfiwm_*.nc .
ln -s /storage/silver/cpom/rk901985/SI3RACC/Downloads/NEMO/SAS_v4.2_RC_FULL/* .
\end{verbatim}

\noindent{}Do not miss off the dots at the end.


\subsection{Test Run}
\label{sec:sas:subsec:test-run}

We shall first make a test run of the SAS using the contents of \verb|EXP00| unchanged.
Create the run directory by giving the following command in the directory \verb|cfgs/UOR_ORCA2_SAS_ICE| under the top-level source directory:

\begin{verbatim}
  cp -a EXP00 4
\end{verbatim}

\noindent{}Change to the run directory \verb|4| and copy the job script \verb|doit4| from \verb|SI3RACC/|\linebreak\verb|Experiments/4|.
The test run specified by \verb|namelist_cfg| (inherited from the reference configuration \verb|ORCA2_SAS_ICE|) is very short, lasting only 6.25 days of simulation time.
The job script \verb|doit4| has an accordingly reduced time limit of 30 minutes, which should be ample.
If you would like to run a longer test, increase the value of the namelist variable \verb|nn_itend| in namelist group \verb|namrun| of \verb|namelist_cfg|.
The units are timesteps of 90 minutes.
A value of 5840 will give a one-year run.
The time limit in \verb|doit4| should be increased too (in experiments 1 and 2 we used a time limit of 16 hours for a one-year run).

When \verb|doit4| is ready (either the original or your modified version), submit it to the batch system with the command

\begin{verbatim}
  sbatch ./doit4
\end{verbatim}

\noindent{}There will be just two main output files:

\begin{verbatim}
  ORCA2_SAS_5d_00010101_00010107_icemod.nc
  ORCA2_SAS_5d_00010101_00010107_grid_T.nc
\end{verbatim}

\noindent{}The first contains sea ice variables, the second ocean variables.
Both contain five-day means.
If you changed the run length, \verb|00010107| be be replaced by something different.
For a one-year run it will be \verb|00011231| (these are dates assuming a start date of 0001-01-01).


\subsection{Three-Year Run}
\label{sec:sas:subsec:3y-run}

We shall now run the SAS for three years with atmospheric forcing from the CORE CIAF data and sea surface forcing from the monthly mean output files from experiment 3.

Start by changing to \verb|cfgs/UOR_ORCA2_SAS_ICE| under the top-level source directory and creating the run directory by copying the template run directory:

\begin{verbatim}
  cp -a EXP00 5
\end{verbatim}

\noindent{}Change to the run directory 5.

As in the three-year run of \verb|UOR_ORCA2_ICE| (Section \ref{sec:forcings}), we need to remove the climatological atmospheric forcing files and replace them with CIAF files:

\begin{verbatim}
  rm *_fill.nc\newline
  ln -s /storage/silver/cpom/rk901985/SI3RACC/Experiments/CIAF4NEMO/*_y200[0-2].nc .
\end{verbatim}

\noindent{}Do not miss off the dot at the end of the second command.

We also need to replace the climatological sea surface forcing files with the files from experiment 3.
The climatological files can be removed with the command

\begin{verbatim}
  rm sas_grid_?.nc
\end{verbatim}

\noindent{}Replacing them with the output files from experiment 3 is slightly less straightforward because the names of output files from NEMO-\SIcu{} are not in the format required for input files.
We saw in Section \ref{sec:forcings} that input files must have names ending \verb|_yYYYY.nc| where \verb|YYYY| is the four-digit year.
Output files, on the other hand, have names ending \verb|_YYYY-YYYY.nc|.
(This is not as perverse as it seems at first because output files can cover more than one year, and so the two \verb|YYYY|'s in their names could be different.)
The renaming necessary so that the output files can be used as input files is handled by the script \verb|ln-nemo| in \verb|SI3RACC/bin|.
This script creates symbolic links in the current directory to the files given on the command line.
If a given file has a name ending \verb|_YYYY-YYYY.nc|, the link name changes this to \verb|_yYYYY.nc|.
Thus we can create links to the required monthly mean output files from experiment 3 with the following command:

\begin{verbatim}
/storage/silver/cpom/rk901985/SI3RACC/bin/ln-nemo ../../UOR_ORCA2_ICE/3/ORCA2_1m*grid*
\end{verbatim}

\noindent{}If you gave the configuration with the full ocean model a name other than \verb|UOR_ORCA2_ICE| or the experiment a name other than \verb|3|, you will have to make the appropriate substitutions in the above command.

Now copy all the files under \verb|SI3RACC/Experiments/5| to the run directory.
The following is a list of the files with their modifications relative to experiment 4 (which used the template run directory \verb|EXP00| unchanged):

\begin{description}

    \item[\texttt{namelist\_cfg}]
        As in experiment 3 (Section \ref{sec:forcings}), namelist group \verb|namrun| specifies a three-year run starting on 2000-01-01, and namelist group \verb|namsbc_blk| specifies the CIAF atmospheric forcing files.
    
        In experiment 2 (Section \ref{sec:mods:subsec:running}), we reduced namelist variable \verb|nn_fsbc| in namelist group \verb|namsbc| from 4 to 1.
        This is not necessary in the current experiment because \verb|nn_fsbc| was already 1 in experiment 4.
    
        In namelist group \verb|namsbc_sas|, the sea surface forcing files from experiment 3 are specified.
        The variables involved represent sea surface velocity (\verb|sn_usp|, \verb|sn_vsp|), temperature (\verb|sn_tem|), salinity (\verb|sn_sal|), and height (\verb|sn_ssh|).
        The other \verb|sn| variables in this group (\verb|sn_e3t| and \verb|sn_frq|) are not used in this experiment.

    \item[\texttt{namelist\_ice\_cfg}]
        As in experiment 2, namelist \verb|namthd_pnd| activates the topographic melt pond scheme and all its local modifications, and namelist group \verb|namalb| turns off the reduction of albedo for thin ice.

    \item[\texttt{file\_def\_nemo-oce.xml}]
        As in experiment 2, useful heat flux variables have been added.
        As in experiment 3, the output files have been split into separate files for each simulation year.

    \item[\texttt{file\_def\_nemo-ice.xml}]
        As in experiment 2, a full set of melt pond variables, ice albedo, and useful heat and mass flux variables have been added.
        As in experiment 3, the output files have been split into separate files for each simulation year.

    \item[\texttt{doit5}]
        This is the same as \verb|doit3| and retains the 48-hour time limit.
        The SAS should run more quickly than the full ocean model, and indeed in the past runs of \SIcu{} with the SAS have taken less than half the time of runs with the full ocean model.
        However, at the time of writing, run times on the RACC have been highly variable, even for experiments of similar type.
        This variability has been swamping the systematic advantage of the SAS, although it still might exist on average.
        It is for this reason that \verb|doit4| has the same time limit as \verb|doit3|.

\end{description}

Submit the job to the batch system using the usual command:

\begin{verbatim}
  sbatch ./doit5
\end{verbatim}

\noindent{}The main output files for 2000 will be the following:

\begin{verbatim}
  ORCA2_SAS_5d_20000101_20021231_icemod_2000-2000.nc
  ORCA2_SAS_5d_20000101_20021231_grid_T_2000-2000.nc
\end{verbatim}

\noindent{}These correspond to the two files from experiment 4 listed in Section~\ref{sec:sas:subsec:test-run}.
There will be similar files for 2001 and 2002.
