\section{Compiling XIOS}
\label{sec:xios}

This appendix describes how the XIOS library under \verb|SI3RACC/Software/xios-trunk| was compiled.
It can be used as a starting point should it ever be necessary to recompile XIOS.

The recommended version of XIOS for use with NEMO-\SIcu{} used to be XIOS~2.5, but since the 4.2.0 release NEMO-\SIcu{} has used features of XIOS that are not in XIOS~2.5.
It is now recommended to download and compile XIOS from the trunk of its Subversion repository.
This is what was done.

The first step was to download the XIOS source code from the Subversion repository by giving the following command under \verb|SI3RACC/Software|:

\begin{verbatim}
  svn co http://forge.ipsl.jussieu.fr/ioserver/svn/XIOS/trunk xios-trunk
\end{verbatim}

\noindent{}At the time of downloading, the Subversion repository was at revision 2331.
In what follows, \verb|SI3RACC/Software/xios-trunk| will be referred to as the \textit{top-level XIOS source directory}.

XIOS is compiled using the script \verb|make_xios| in the top-level XIOS source directory.
Platform-specific compiler options are specified in configuration files in the \verb|arch| subdirectory.
Unlike NEMO-\SIcu{}, which uses just a single configuration file, XIOS uses three files with extensions \verb|.env|, \verb|.fcm|, and \verb|.path|.
Configuration files for compiling XIOS on the RACC have been created with names beginning \verb|arch-GCC_RACC| and can be found under \verb|SI3RACC/Software/|\linebreak\verb|xios-mods-trunk/arch|.
Symbolic links to these files were placed in the \verb|arch| subdirectory of the top-level XIOS source directory.

Unlike with NEMO-\SIcu{}, it is not necessary to load any modules before compiling XIOS.
The necessary module commands are instead included in the \verb|.env| file.

XIOS was compiled using the following command in the top-level XIOS source directory:

\begin{verbatim}
  ./make_xios --arch GCC_RACC
\end{verbatim}

\noindent{}The output from this command has been saved in the file \verb|make_xios-out|.
