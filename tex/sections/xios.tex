\section{Downloading and compiling XIOS}
\label{sec:xios}

The third-party software XIOS (XML Input/Output Server) is required to run NEMO and must be downloaded and compiled as a prerequisite.\footnote{
    Strictly speaking, it is only required when running NEMO as a distributed memory job, i.e., in parallel across multiple CPUs, but we almost always do this.
}
It handles the reading of input forcing files into the model and the selection and writing of output diagnostics during model runs.
Currently, there is a version 2 and a version 3 of XIOS, maintained in parallel.
The latest version of NEMO, version 5, is compatible with both, with XIOS3 supposedly leading to more efficient I/O and hence faster simulation run times.
Alas, we are yet to get XIOS3 working on RACC2, so for the time being, we stick with XIOS2.

The first step is to download the XIOS source code from its Subversion repository.
We download the trunk as recommended by the NEMO developers.

\begin{enumerate}

    \item Change to the directory in which you wish to download and compile the XIOS source code.
        It can be anywhere; in this example we create a directory called \verb|software| under the user's home directory (\verb|$HOME|):

\begin{verbatim}
  $ mkdir -p $HOME/software
  $ cd $HOME/software
\end{verbatim}

    \item Give the following command to download XIOS2 to a new subdirectory called \verb|XIOS2|:

\begin{verbatim}
  $ svn co http://forge.ipsl.jussieu.fr/ioserver/svn/XIOS2/trunk XIOS2
\end{verbatim}

    It may take a few minutes to download.

\end{enumerate}

In the top-level XIOS source directory (\verb|$HOME/software/XIOS2|, in this case), there is a script called \verb|make_xios| (no extension) which is used to compile it.
Platform-specific compiler options and library paths are specified in configuration files in the \verb|arch| subdirectory, all of which must start with \verb|arch-|.
For each any given platform, there are three such configuration files with extensions \verb|.env|, \verb|.fcm|, and \verb|.path|.
There are templates for various cases but a custom set of configuration files is required to compile on the RACC2.
These are provided in the auxiliary resources under \verb|xios/xios2/arch|, called \verb|arch-racc2.env|, \verb|arch-racc2.fcm|, and \verb|arch-racc2.path|.

\begin{enumerate}\setcounter{enumi}{2}
    \item Copy these three files into your \verb|arch| subdirectory.
    \item Give the following command from the top-level XIOS source directory to compile XIOS:
\begin{verbatim}
  $ ./make_xios --arch racc2
\end{verbatim}
\end{enumerate}

In the above command, the value given to the argument \verb|--arch| is the part of the configuration file names without their prefix \verb|arch-| and without their extension, which in this case is just \verb|racc2|.
Compilation takes approximately 15~minutes to complete, assuming it does so successfully.
